\documentclass{article}

\usepackage[margin=1.5in]{geometry}
\usepackage{upgreek}
\usepackage{lscape}
\usepackage{rotating}
\usepackage{amsmath}
\usepackage{tikz}
% \usepackage{graphicx}
\usepackage{graphics}
\usepackage{caption}
\usepackage{subcaption}
\usepackage{mathrsfs}
\usepackage[toc,page]{appendix}

%Section style
\usepackage{etoolbox} %for configuration of sloppy
\usepackage{xcolor}


\definecolor{secnum}{RGB}{102,102,102}

\makeatletter
    \def\@seccntformat#1{\llap{\color{secnum}\csname the#1\endcsname\hskip 16pt}}
\makeatother
%end section style

\begin{document}

\begin{center}
\textsc{\Large Visualization}\\[0.5cm]
\textsc{\large "Midway"-report}\\[0.5cm]
\textsc{\large Lasse Ahlbech Madsen, Simon Maibom}\\[0.5cm]
\vspace{1 cm}
\end{center}
% \tableofcontents

\section{Introduction}


\section{Task and dataset}
The target audience of the project is intermediate Settlers of Catan players, who wants to get better at the game by getting a deeper understanding of how to place the two starting towns in order to maximize the chance to win. In Settlers, how you choose to place the first two towns has a large impact on how the game plays out in you're favor.
The dataset \footnote{https://www.kaggle.com/lumins/settlers-of-catan-games} consist of data over 50 played games, each game has 4 lines that consist of starting position, points gained, placements of starting towns, total resource gains and losses from production, robber cards and trade.

\section{Design}


\section{Implementation}

\end{document}
